% This file should be replaced with your file with an appendices (headings below are examples only)

% Placing of table of contents of the memory media here should be consulted with a supervisor
%\chapter{Contents of the included storage media}

%\chapter{Manual}

%\chapter{Configuration file}

%\chapter{Scheme of RelaxNG configuration file}

%\chapter{Poster}

\chapter{Acronyms}

Since there are many used acronyms in this thesis, it is appropriate to list them and explain them
for better understanding.

\begin{itemize}
  \item API - Application Programming Interface - interface for use of different project
  \item DNF - Dandified YUM - Successor of the YUM package manager
  \item RPM - Red Hat Package Manager - Low-level RPM package manager 
  \item RPQR - RPM Package Query Resolver - Project created during this thesis
  \item YUM - The Yellowdog Updater, Modified - High-level RPM package manager
  \item GPL - General Public License - Open source license
\end{itemize}

\chapter{Installation}

There is a description of how to install the RPQR project either from a source or an RPM package.


Installation from the source can be performed with \textbf{pip install .} command executed from the
the root of the project source.


Installation of the RPRQ projects' RPM package has to be done with help of the Fedora Copr repository.
Before you can use the RPQR project, you need to install the python3-dnf package from your
systems' official repositories. This is necessary, because DNF API is not distributed through Python
package index.

\textbf{\#dnf copr enable tkorbar/RPQR; dnf install -y python3-RPQR}


If you do not wish to receive updates or install RPQR in the future then you need to remove the
repository configuration from your system.


\textbf{\#dnf copr disable tkorbar/RPQR}

